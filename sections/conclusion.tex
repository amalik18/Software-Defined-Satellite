\documentclass[../main.tex]{subfiles}
\usepackage[utf8]{inputenc}
\usepackage{blindtext}
\usepackage{float}
\usepackage{graphicx}
\usepackage{siunitx}

\begin{document}

In conclusion, the goal of this paper was to dive into the specifics of Software-Defined Satellites (SDSs) and to discuss what separates SDSs from regular satellites. We determined the difference between the two being the radios, a software-defined radio as opposed to a static radio.

We discussed the pros and cons of having an SDR and how it affects satellite communications. We identified a major pro, that being the programmability of the SDR post launch which allows satellites to change missions on the spot. Rather than building mission-specific satellites, we can now build more general purpose satellites that can be used to accomplish many missions.

We also identified some applications of SDSs. One of the more popular applications of SDSs are a Satellite Software-Defined Network (SDN), we examined the overall design of an SDN and how an SDS could help provide more networking capabilities in space.

We also addressed some big players in the field and the platforms provided by them. In addition, we evaluated opportunities that currently exist in the field and the gaping hole that is currently missing. 

In closing, SDSs are steadily becoming a pillar in the satellite communications realm, as well as, slowly replacing fixed-state satellites. By switching to a more modular and modifiable approach, resources can be fully utilized, missions can be programmed and changeable, and lastly we'll move towards a more efficient satellite lifecycle.

\end{document}