\documentclass[../main.tex]{subfiles}

\begin{document}

Satellites have evolved immensely in the last 20 years. They've become ubiquitous in recent years, from satellite television to weather forecasting, satellites are everywhere. The military, especially, relies heavily on space information systems (SIS) from information reconnaissance, information transmission, to information countermeasures[1]. 

There is a massive drawback to most SISs...their flexibility. Most satellites are designed for a specific function and they can be categorized as such: communication satellites, weather satellites, navigation satellites, etc. Once these satellites are launched, their orbit, capabilities, functions, and operations are all static, fixed, unchangeable[1]. To put into perspective, expensive satellites such as communication satellites can only be used for communications, not communication reconnaissance. The many GPS satellites can only provide positioning and timing services, they cannot be used for broadcasting[1]. 

It can be concluded that the current norm of space information systems brings a few major concerns. Firstly, a lack of flexibility, the fixed nature of satellite missions cannot meet the ever-changing user demand. Secondly, in order to realize a complex, multi-task SIS, different satellites are needed to address each unique functionality. Lastly, the hardware-centric design of the payload is not configurable and does not adapt to environmental changes.

Small satellites in clusters or constellations have become increasingly popular as a means to conduct scientific and technological missions in a more affordable manner. As the complexity of these missions grow, the issues with increasing communication windows, supporting multiple signals, and increasing data rates become much more apparent. As a means to resolve these issues software abstractions become more and more valuable \cite{sdr_satellite}.

\end{document}